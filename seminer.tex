\documentclass[a4paper,12pt]{article}
\usepackage[top=3cm,bottom=2.5cm,left=3.5cm,right=2.5cm]{geometry}
\usepackage{ucs}
\usepackage[utf8x]{inputenc}
\usepackage[T1]{fontenc} 
\usepackage{amsmath}
\usepackage{amsfonts}
\usepackage{amsthm}
\usepackage{amssymb}
\usepackage{setspace}
\usepackage{mathptmx}
\usepackage{titlesec}
\usepackage{enumitem}
\linespread{1.5}
\numberwithin{equation}{section}
%\usepackage[lflt]{floatflt}
\renewcommand{\labelenumi}{(\roman{enumi})}
%\voffset	=-10.0mm
%\hoffset	= 0.96cm
%\textheight	=614pt
%\footskip	=-40mm
%\marginparwidth = 0mm
%\evensidemargin	=0mm
%\marginparsep	= 0mm
%\topmargin	=30mm	
%\oddsidemargin	=-10.mm		
%\evensidemargin	=0mm		
%\headheight	=0.mm
%\headsep	=0.mm
%\textheight	=220.mm
%\textwidth	=150.mm
\title{Ders Notları}
\date{}
\def\chaptername{Bölüm}
\newtheoremstyle{italik}{}{}{\normalfont}{}{\bfseries}{}{ }{} % basic definitions, use roman font: 
\theoremstyle{italik}
\newtheorem{ornek}{Örnek}[part]
\newtheorem{teorem}{Teorem}[section]
\newtheorem{lemma}[teorem]{Yardımcı Teorem}
\newtheorem{tanim}{Tanım}[section]
\newtheorem*{ispat}{İspat}
\newtheorem*{cozum}{Çözüm}

\titleformat{\section}{
  \titlerule \vspace{.8ex}% 
  \large\bfseries}
{\thesection.}{.5em}{}
\begin{document}

\abovedisplayshortskip=0pt
\belowdisplayshortskip=0pt
\belowdisplayskip=0pt
\abovedisplayskip=0pt


\title{Some Unrelated Topics of Interest}
\author{Michael Trott}
\date{\today}
\maketitle
\begin{abstract}
Bu çalışmada, küçük radikale sahip E-tamlanmış tam modüler L kafesinin zayıf tümlenmiş olması için gerek ve yeter koşulun 
$L$ nin yarıyerel olmasıdır. $L$ nin bir dual sonlu zayıf tümlenmiş kafes olması için gerek ve yeter koşul $L$ nin her maksimal elemanının 
$L$ içinde bir zayıf tümleyene sahip olmasıdır. $ a/0 $ bir zzayıf tümlanmiş altkafes ve $ 1/a $ altkafesi mazimal elemana sahip değilse
$L$ zayıf tümlenmiştir.
\end{abstract}
\section{Giriş}
Bu çalışma boyunca $L$, en küçük elemanı $0$ ve en büyük elemanı $1$ olan tam modüler kafesi temsil edecektir. Kafes olarak da tam modüler kafesleri düşüneceğiz. 
\begin{tanim}
Bir $a \in L $ elemanı verilsin. $ L $ deki her $ b \neq 1 $ elemanı için $ a \vee b \neq 1 $ ise $a$ elemanına $L$ içinde küçüktür denir
ve $ a\ll L $ şeklinde gösterilir.
\end{tanim}
Bu tanıma denk olarak aşğıdaki tanım da verilebilir.
\begin{tanim}
 $a$ nın $L$ içinde küçük olması için gerek ve yeter koşul $a \vee b = 1$ olan her $b$ için $b=1$ olmasıdır.
\end{tanim}
\begin{tanim}
 Bir $b \in L $ elemanı verilsin. $ a \vee b=1 $ ve $a$ bu özelliğe göre minimal ise $a$ ya $b$ nin $L$ içinde bir tümleyeni denir.
\end{tanim}
Aşasğıdaki tanım da bu tanıma denktir.
\begin{tanim}
$a$ nın $b$ nin $L$ içinde bir tümleyeni olması için gerek ve yeter koşul $a \vee b =1 $ ve $ a \wedge b \ll a/0 $ olmasıdır.
\end{tanim}
Son koşulu zayıflatırsak zayıf tümlenmişlik tanımını elde ederiz.
\begin{tanim}
$a$, $b$ nin $L$ içinde bir zayıf tümleyenidir ancak ve ancak $a \vee b =1 $ ve $ a \wedge b \ll L $ dir.
\end{tanim}
$L$ nin her elemanı $L$ içinde bir tümleyene(zayıf tümleyene) sahipse $L$ ye tümlenmiştir(zayıf tümlenmiştir) denir.
\begin{tanim}
Bir $a \in L $ elemanı verilsin. $ L $ deki her $ b \neq 0 $ elemanı için $ a \wedge b \neq 0 $ ise $a$ elemanına $L$ içinde büyüktür denir
ve $ a \trianglelefteq L $ şeklinde gösterilir.
\end{tanim}
\begin{tanim}
 Bir $a \in L $ elemanı verilsin. Bir $b \in L$ elemanı için, $ a \wedge b = 0 $ ve $ a \vee b \trianglelefteq L $ ise $b$ ye
$a$ nın \emph{E-tamlayıcısı} denir. Bir $L$ kafesinin her elemanı $L$ içinde bir E-tamlayıcıya sahipse $L$ kafesine \emph{E-tamlanmıştır} denir.
\end{tanim}
\begin{tanim}
 $L$ nin her $a$ elemanı için, $ a \vee b = 1 $ ve $ a \wedge b = 0 $ olacak şekilde bir $ b \in L $ elemanı varsa $L$ ye tamlanmıştır denir.
\end{tanim}
\begin{tanim}
 $L$ nin bütün maksimal elemanlarının en büyük alt sınırı $L$ nin radikali olarak tanımlanır ve $rad(L)$ ile gösterilir. Eğer $ 1/rad(L) $
tamlanmış ise $L$ ye \emph{yarıyerel} kafestir denir.
\end{tanim}
\section{Zayıf Tümlenmiş Kafesler}
\begin{lemma}\label{thm:teo1}
$ a,b \in L $ ve $ a < b $ olsun. Bu durumda aşağıdakiler sağlanır.
\begin{enumerate}

\itemsep 0em
\item $ a \ll b/0 $ ise her $ c \in L $ için $ a \vee c \ll (b \vee c)/c $ olur. \label{teo11}
\item $ b \ll L $ olması için gerek ve yeter koşul $ a \ll L $ ve $ b \ll 1/a $ olmasıdır. \label{teo12}
\item $ a \ll b/0 $ ise $ a \ll L $ dir. \label{teo13}
\end{enumerate}
\begin{ispat}
\setlength{\leftmargini}{0pt}
\begin{enumerate}
\itemindent 25pt
 \item 


 $c \in L $ ve $ t \in (b \vee c)/c  $ için $ (a \vee c ) \vee t = 1_{(b \vee c)/c} = b \vee c $ olsun. 
$ t \in (b \vee c)/c $ olduğundan $ c \leq t $ dir. Bu durumda $ a \vee t = a \vee (c \vee t) = (a \vee c ) \vee t = b \vee c $ 
elde edilir. $ a \vee t = b \vee c $ eşitliğinin iki tarafı, sağdan $b$ ile $\wedge$ işlemine tabi tutulursa 
$(a \vee t) \wedge b = (b \vee c) \wedge b = b $ elde edilir.
 $ a < b $ olduğundan modülerlik kuralı gereği $ a \vee (t \wedge b) = (a \vee t) \wedge b = b $ bulunur. 
$ t \wedge b \leq b $ olduğundan $ t \wedge b \in b/0 $ dır. Bu durumda $ a \ll b/0 $ ve 
$ a \vee (t \wedge b)=b=1_{b/0} $ olduğu dikkate alınırsa $ b \wedge t = 1_{b/0} = b $ bulunur. 
Bu ise $ b \leq t $ olmasını gerektirir. O halde $ b \vee c \leq t \vee c = t $ olur. Aynı zamanda 
$ t \in (b \vee c)/c $ olduğundan $ t \leq b \vee c $ dir. Buradan $ t = b \vee c $ bulunur. Böylece 
$ a \vee c \ll (b \vee c)/c $ bulunur.
\item $ (\Rightarrow :) $ \ \ 
$ t \in L $, $ b \ll L $ ve $ a \vee t = 1 $ olsun. Buradan $ b \vee (a \vee t ) = b \vee 1 = 1 $ yazılabilir. 
$ a < b $ olduğu dikkate alınırsa $ b \vee t = ( b \vee a ) \vee t = b \vee (a \vee t ) = 1 $ elde edilir. 
$ b \ll L $ olduğundan $ t = 1 $ dir. O halde $ a \ll L $ dir. \\
$ s \in 1/a $ ve $ b \vee s = 1 $ olsun. $ s \in 1/a $ olduğundan $ a \leq s $ dir ve $ a \vee s = s $ eşitliği 
vardır. O halde $ b \vee (a \vee s) = b \vee s = 1 $ dır ve $ b \ll L $ olduğundan $ a \vee s = 1 $ bulunur. 
Diğer taraftan $ a \tt L $ olduğu da dikkate alınırsa $ s= 1 $ bulunur. O halde $ b \ll 1/a $ dır. \\
$ (\Leftarrow :) $ \quad $ a \ll L $, $ b \ll 1/a $ ve $ t \in L $ için $ b \vee t = 1 $ olsun. Burada eşitliğin 
her iki tarafına sağdan $ a $ elemanı ile $ \vee $ işlemi uygulanırsa $ ( b \vee t ) \vee a = 1 \vee a = 1 $ bulunur. 
$ b \vee (t \vee a) = ( b \vee t ) \vee a $ eşitliğinden $ b \vee (t \vee a) = 1 $ elde edilir. 
$ t \vee a \in 1/a $ ve $ b \ll 1/a $ olduğundan $ t \vee a = 1 $ dir. $ a \ll L $ olduğunu da 
düşünürsek $ t = 1 $ bulunur. O halde $ b \ll L $ dir.
\item $ a \ll b/0 $ ve $ t \in L $ için $ a\vee t = 1 $ olsun. Eşitliğin her iki tarafına sağdan $ b $ ile $ \wedge $ 
işlemi uygulanırsa $ (a \vee t) \wedge b = 1 \wedge b = b $ eşitliği elde edilir. $ a < b $ olduğundan 
modüler kuralı gereği $ a \vee (t \wedge b) = a \vee (t \wedge b) = b $ bulunur. O halde $ a \ll b/0 $ olduğundan 
$ t \wedge b = 1 $ yani $ b \leq t $ dir. Buradan $ t = b \vee t = (a \vee b) \vee t = a \vee (b \vee t) = a \vee t = 1 $ 
bulunur. $ a \ll L $ dir.
\end{enumerate}
\end{ispat}
\end{lemma}
\begin{teorem}\label{thm:teo2}
$ L $ zayıf tümlenmiş bir kafes ise, her $ a \in L $ için $ 1/a $ altkafesi de zayıf tümlenmiştir.
\end{teorem}

\begin{ispat}
 $ b \in 1/a $ olsun. $ L $ zayıf tümlenmiş olduğundan $ b $ nin $ L $ içinde $ x $ zayıf tümleyeni vardır. O halde $ x \vee b = 1 $ 
ve $ x \wedge b \ll L = 1/0 $ olur. $ ( a \vee x ) \vee b = 1 $ olduğu açıktır. Yardımcı Teorem~\ref{thm:teo1} (\ref{teo11}) ve $ a \leq b $ 
olduğu düşünülürse, modülerlik kuralı gereği $ ( a \vee x ) \wedge b = ( b \wedge x ) \vee a \ll ( 1 \vee a )/a = 1/a $ elde edilir. 
O halde $ 1/a $ zayıf tümlenmiştir.
\end{ispat}

\begin{teorem}\label{thm:teo3}
 $ 1/a $, $ L $'nin zayıf tümlenmiş bir altkafesi olacak şekilde bir $ a \ll L $ elemanı varsa $ L $'de zayıf tümlenmiştir.
\end{teorem}
\begin{ispat}
 $ x \in L $ olsun. Bu durumda $ x \vee a \in 1/a $ olur ve $ 1/a $ zayıf tümlenmiş olduğundan $ x \vee a $ nın $ 1/a $ içinde 
bir $ y $ zayıf tümleyeni vardır. O halde $ y \vee ( x \vee a ) = 1 $ ve $ y \wedge ( x \vee a ) \ll 1/a $ olur. 
Bu durumda $ y \in 1/a $ olduğundan $ a \leq y $ dir ve $ y \vee x = ( y \vee a ) \vee x = y \vee ( x \vee a ) = 1 $ bulunur. 
Yardımcı Teorem~\ref{thm:teo1} (\ref{teo12}) gereği $ y \wedge ( x \vee a ) \ll L $ dir. 
$ y \wedge x \leq y \wedge ( x \vee a ) \ll L $ olduğundan $ y \wedge x \ll L $ elde edilir. Böylece $ y $, $ x $ in $ L $ içinde 
bir zayıf tümleyeni olur. 
\end{ispat}
\begin{teorem}\label{thm:teo4}
 $ a $, zayıf tümlenmiş bir $ L $ kafesindeki bir elemanın tümleyeni ise $ a/0 $ bölüm altkafeside zayıf tümlenmiştir.
\end{teorem}
\begin{ispat}
 $ a $, $ b $'nin $ L $ içinde bir zayıf tümleyeni olsun. Dolayısıyla $ a \vee b = 1 $ ve $ a \wedge b \ll a/0 $ dir. Bu durumda 
$ 1/b = ( a \vee b ) /b \cong a / ( a \wedge b ) $ olup(Bunu ekle) Teorem~\ref{thm:teo2} gereği $ a / ( a \wedge b ) $ zayıf 
tümlenmiştir.  Teorem~\ref{thm:teo3}, $ a/0 $ bölüm altkafesinde düşünülürse $ a/0 $'ın zayıf tümlenmiş olduğu elde edilir.
\end{ispat}
\begin{teorem}
 $ b $, $ a $ nın $ L $ içinde bir zayıf tümleyeni ve $ c \ll L $ ise $ b $, $ a \vee c $ nin $ L $ içinde bir zayıf tümleyeni olur.
\end{teorem}
\begin{ispat}
\noindent $ b $, $ a $ nın $ L $ içinde bir zayıf tümleyeni olduğundan $ a \vee b = 1 $ ve $ a \wedge b \ll L $ dir. Bu durumda 
$ ( a \vee c ) \vee b = ( c \vee a ) \vee b = c \vee ( a \vee b ) = 1 $ olduğu açıktır. $ d = a \wedge b $ ve $ u = ( a \vee c ) \wedge b $ 
diyelim. $ y \in L $ ve $ u \vee y = 1 $ olsun. $ x = y \vee d $ dersek $ u \vee x = u \vee ( y \vee d ) = 1 $ bulunur. Bu durumda 
$ b = b \wedge 1 = b \wedge ( u \vee x ) = u \vee ( b \wedge x ) $ ve 
$ 1 = a \vee b = a \vee u \vee ( b \wedge x ) = a \vee [ ( a \vee c ) \wedge b ] \vee ( b \wedge x ) $ olur. $ a \leq a \vee c $ 
olduğundan modüler kuralı gereği 
\begin{equation}\label{esitlik1}
 [ (a \vee c )  \wedge ( a \vee b ) ] \vee ( b \wedge x ) = a \vee [ ( a \vee c ) \wedge b ] \vee ( b \wedge x )  =  1
\end{equation}
elde edilir. Diğer taraftan
\begin{equation}\label{esitlik2}
[ (a \vee c )  \wedge ( a \vee b ) ] \vee ( b \wedge x ) = [(a \vee c ) \wedge 1 ] \vee ( b \wedge x ) = a \vee c \vee ( b \wedge x )
\end{equation}
eşitliği vardır. \eqref{esitlik1} ve \eqref{esitlik2} eşitliklerinden $ c \vee a \vee ( b \wedge x ) = 1 $ bulunur. $ c \ll L $ olduğundan $ a \vee ( b \wedge x ) = 1 $ dır. 
Eşitliğin her iki tarafına soldan $ b $ ile $ \wedge $ işlemi uygulanırsa 
\begin{equation}\label{esitlik3}
 b = b \wedge [ a \vee ( b \wedge x ) ] = ( b \wedge a ) \vee ( b \wedge x)
\end{equation}
elde edilir. Ayrıca 
\[ b \wedge x = b \wedge ( y \vee d ) =  b \wedge ( y \vee ( a \wedge b ) = ( b \wedge y ) \vee (a \wedge b ) \]
olduğundan $ a \wedge b \leq b \wedge x $ yani 
\begin{equation}\label{esitlik4}
 ( b \wedge a )\vee ( b \wedge x ) = b \wedge x
\end{equation}
bulunur. \eqref{esitlik3} ve \eqref{esitlik4} eşitliklerinden $ b = b \wedge x $ yani $ b \leq x $ olur. Bu durumda  
$ 1 = u \vee x \leq b \vee x = x $ olup $ x = 1 $ elde edilir. $ d = a \wedge b \ll L $ ve $ y \vee d = x = 1 $ olduğundan 
$ y = 1 $ bulunur. Böylece $ b $, $ a \vee c $ nin $ L $ içinde bir zayıf tümleyenidir.
\end{ispat}




















\begin{teorem}
 $ a $, $ L $ içinde bir zayıf tümleyene sahip ve $ 1/a $ ve $ a/0 $ zayıf tümlenmiş ise $ L $ de zayıf tümlenmiştir.
\end{teorem}
\begin{ispat}
 $ b $, $ a $ nın $ L $ içinde bir zayıf tümleyeni olsun. O halde $ a \vee b = 1 $ ve $ a \wedge b \ll L $ olur. 
$ a/0 $ zayıf tümlenmiş ve $ a \wedge b \in a/0 $ olduğundan teorem gereği $ a / a \wedge b $ de zayıf tümlenmiştir. 
Diğer taraftan $ b/(a \wedge b) \cong (a \vee b)/a = 1/a $ olduğundan $ b/(a \wedge b) $ zayıf tümlenmiştir. 
$ a/(a \wedge b) $ ve $ b/(a \wedge b) $, $ 1/(a \wedge b) $ nin zayıf tümlenmiş bölüm alt kafesleri ve $ a \vee b = 1 $ olduğundan 
önerme gereği $ 1/(a \wedge b) $ de zayıf tümlenmiştir. 
$ a \wedge b \ll L $ ve $ 1/(a \wedge b) $ zayıf tümlenmiş ise önerme gereği $ L $ zayıf tümlenmiştir.
\end{ispat}
\begin{teorem}
 $ L $ tamlanmış ise her $ a \in L $ için $ a/0 $ tamlanmıştır.
\end{teorem}
\begin{ispat}
 $ x \in a/0 $ olsun. $ L $ tamlanmış olduğundan $ x \vee y = 1 $ ve $ x \wedge y = 0 $ olacak şekilde bir $ y \in L $ vardır 
ve $ a \wedge y \leq a $ olduğundan ise $ a \wedge y \in a/0 $ dır. Bu durumda $ x \wedge (a \wedge y) = a \wedge (x \wedge y) = 0 $ olduğu açıktır. 
Modüler kuralından
\[
 x \vee (a \wedge y) =  a \wedge (x \vee y) = a \wedge 1 = a = 1_{a/0}
\]
elde edilir. Dolayısıyla $ x $, $ a/0 $ içinde $ a \wedge y $ nin bir tamlayanıdır. O halde $ a/0 $ tamlanmıştır.

\end{ispat}
\begin{lemma}
 $a$, $ L $ içinde büyük ise her $ b \in L $ için $a \wedge b$, $b/0$ içinde büyüktür.
\end{lemma}
\begin{ispat}
 $ c \in b/0 $ ve $(a \wedge b) \wedge c=0$ olsun. $ c \in b/0 $ olduğundan $ b \wedge c = c $ dir. 
 $a \wedge (b \wedge c) =(a \wedge b) \wedge c=0$ ve $a \trianglelefteq L $ olduğundan $c = b \wedge c = 0$ olur. 
O halde $a \wedge b$, $b/0$ içinde büyüktür.

\end{ispat}






\end{document}
